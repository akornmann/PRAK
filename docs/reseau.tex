\documentclass[10pt,a4paper]{article}

\usepackage[utf8]{inputenc}
\usepackage[T1]{fontenc}
\usepackage[french]{babel}

%format de la page%
\usepackage{geometry}
\geometry{hmargin=2.5cm,vmargin=2.5cm}

%style des headers/footers%
\usepackage{lastpage}
\usepackage{fancyhdr}
\pagestyle{fancy}

\renewcommand{\headrulewidth}{1pt}
\fancyhead[C]{} 
\fancyhead[L]{\leftmark}
\fancyhead[R]{}

\renewcommand{\footrulewidth}{1pt}
\fancyfoot[C]{\textbf{\thepage/\pageref{LastPage}}} 
\fancyfoot[L]{Réseau}
\fancyfoot[R]{Alexandre Kornmann}

%packages maths%
\usepackage{amsmath}
\usepackage{amsfonts}
\usepackage{amssymb}

\newcommand{\N}{{\mathbb{N}}}
\newcommand{\Z}{{\mathbb{Z}}}
\newcommand{\R}{{\mathbb{R}}}


%affichage de code source%
\usepackage{listings}

\lstset{
language=fortran,
frame=single,
keywordstyle=\color{blue}\bfseries,
commentstyle=\color{green}\itshape,
stringstyle=\color{red},
showstringspaces=false,
breaklines=true,
breakatwhitespace=true}

%divers%
\usepackage{caption}
\usepackage{url}
\usepackage{subfig}
\usepackage[colorlinks=true,urlcolor=blue]{hyperref}
\usepackage{float}
\usepackage{graphicx}


\title{Projet réseau}
\author{Alexandre Kornmann}

\begin{document}

\maketitle

\newpage

\tableofcontents

\newpage

\section{Documentation utilisateur}
\subsection{Démarage d'un client}

Le client démarre par la commande
\begin{center}
\fbox{./PRAK client} 
\end{center}

\subsection{Liste des commandes clients}

\subsubsection{catalogue}
\begin{center}
\fbox{catalogue} 
\end{center}
\textbf{catalogue} (ou son alias \textbf{lib}) affiche la bibliotèque sur la sortie standard.

\subsubsection{lire}
\begin{center}
\fbox{lire FICHIER} 
\end{center}
\textbf{lire} (ou son alias \textbf{dl}) affiche le fichier FICHIER sur la sortie standard.

\subsubsection{stocker}
\begin{center}
\fbox{stocker FICHIER TITRE} 
\end{center}
\textbf{stocker} (ou son alias \textbf{ul}) ajoute le fichier FICHIER sur deux serveurs, et l'ajoute catalogue sous le titre TITRE. FICHIER doit se trouver dans le dossier où a été lancé l'application client.

\subsubsection{detruire}
\begin{center}
\fbox{detruire NOMFICHIER} 
\end{center}
\textbf{detruire} (ou son alias \textbf{rm}) détruit le fichier FICHIER de tout les serveurs, et le retire du catalogue.

\subsubsection{exit}
\begin{center}
\fbox{exit}
\end{center}
\textbf{exit} quitte le client.

\subsection{Fichier de configuration}

Le format d'une ligne du fichier du configuration qui contient la liste de tout les serveurs à contacter est le suivant :
\begin{center}
\fbox{NOMHOTE|IPV4|IPV6 PORT}
\end{center}
Chaque ligne contient un enregistrement de ce type.\\
\textit{Attention : ne pas laisser de ligne vide à la fin du fichier, ou d'espace inutile en fin de ligne.}

\newpage

\section{Protocoles}

\subsection{Philosophie des protocoles}
Tout les protocoles suivants sont basés sur une philosophie commune : le client demande, le serveur répond.
Le serveur ne peut pas prendre d'initiative dans une transaction, il se contente de réagir aux demandes du client.
\subsection{Format du datagramme}

Le datagramme qui permet la communication est de la forme :
\begin{enumerate}
 \item int : \textbf{code}
 \item int : \textbf{seq}
 \item 512 char : \textbf{data}
\end{enumerate}

Par la suite, on notera un datagramme de la façon suivante :

\begin{center}
 \fbox{dg(code,seq,data)}
\end{center}

Des paramètres peuvent être omis.

Exemples :
\begin{itemize}
 \item dg(0,0,texte)
 \item dg(5,458,)
 \item ...
\end{itemize}


code peut prendre sa valeur entre 0 et 6, ce qui correspond au type de paquet suivants :
\begin{itemize}
 \item 0 = CONNECTRA
 \item 1 = DISCONNECTRA
 \item 2 = DOWNLOAD
 \item 3 = UPLOAD
 \item 4 = ADD
 \item 5 = REMOVE
 \item 6 = GET
\end{itemize}

\subsection{CONNECTRA : Requête de connexion d'un client à un serveur}
Ce protocole très simple permet de signifier à un serveur qu'on souhaite entamer une transaction avec lui.
Le client envoie \textbf{dg(0,0,)}. Le serveur répond \textbf{dg(0,1,)}.

\subsection{DISCONNECTRA : Requête de déconnexion d'un client à un serveur}
Ce protocole très simple permet de signifier à un serveur qu'on souhaite mettre un terme à une transaction avec lui.
Le client envoie \textbf{dg(1,1,)}. Le serveur répond \textbf{dg(1,0,)}.

\subsection{DOWNLOAD : Récupération d'un fichier présent sur le serveur}
Pour initier la transaction, le client envoie le paquet \textbf{dg(2,0,nom du fichier)}.
Les réponses possibles sont :
\begin{itemize}
 \item \textbf{dg(2,2,)} : Le fichier est dans la librairie et est disponible sur le serveur.
 \item \textbf{dg(2,1,)} : Le fichier est dans la librairie masi n'est pas disponible sur le serveur.
 \item \textbf{dg(2,0,)} : Le fichier n'est pas dans la bibliotèque
\end{itemize}

En fonction de cette réponse, le client peut décider de demander le fichier, avec \textbf{dg(2,0,nom du fichier)}
Si le serveur trouve le fichier, il va répondre avec la taille du fichier en numéro de séquence, sinon le serveur répond 0 en numéro de séquence.
Par la suite le client envoie au serveur des paquets de la forme \textbf{dg(2,x,)}, avec \textbf{x} l'indice du dernier caractère du fichier que l'on souhaite recevoir.
La fin du transfert est signifié à la réception d'un paquet contenant moins de 512 caractères.

\subsection{UPLOAD : Envoi d'un fichier sur un serveur}
Pour initier la transaction, le client envoie le paquet \textbf{dg(3,x,)} et recoit en acquittement le paquet \textbf{dg(3,x,)}.
Ensuite, le client envoie trois paquets contenant les métadonnées : \textbf{dg(3,0,nom du fichier)}, \textbf{dg(3,1,titre du fichier)}, et \textbf{dg(3,taille du fichier,)}. Le serveur acquitte les trois en renvoyant \textbf{dg(3,taille du fichier,)}.
Pour le transfert des données, le client envoie un paquet contenant 512 caractères, \textbf{dg(3,seq,chaine de 512 caractères)} que le serveur acquitte par \textbf{dg(3,seq,)}.
Un paquet de longueur inférieur à 512 signifie la fin de la transaction. \textbf{seq} indique la position du dernier caractères de la chaîne dans le fichier.

\subsection{ADD : ajout d'un fichier à la bibliotèque}
Afin d'ajouter un fichier à la bibliotèque, le client envoie deux paquets contenant respectivement le nom du fichier et son titre (\textbf{dg(4,0,nom du fichier)} et \textbf{dg(4,1,titre du fichier)}).

Le serveur acquitte chaque paquet en répondant avec le même numéro de séquence que le paquet recu (c'est à dire \textbf{dg(4,0,)} ou \textbf{dg(4,1,)})

\subsection{REMOVE : retrait d'un fichier de la bibliotèque}
Afin de retirer un fichier de la bibliotèque, le client envoie un paquet \textbf{dg(5,0,nom du fichier)}.
Le serveur acquitte chaque paquet en répondant \textbf{dg(5,0,)}.

\subsection{GET : récupération de l'intégralité de la bibliotèque}
Afin d'initier la transaction, le client envoie un paquet \textbf{dg(6,0,)}.
Le serveur répond alors \textbf{dg(6,lignes,)} avec \textbf{lignes} le nombre d'entrée dans la bibliotèque.
Le client peut alors demander la ième ligne de la bibliotèque (avec \textbf{dg(6,i,)}). Le serveur répond alors en envoyant deux paquets contenant respectivement un nom de fichier et le titre associé (\textbf{dg(6,i,nom du fichier)} et \textbf{dg(6,i+lignes,titre})).

\section{Architecture}


\end{document}

